\section{Reflexivität und schwache Konvergenz}

\subsection{Adjungierter Operator}

\begin{definition}{Adj. Op.}
    Seien \(X,Y\) Br., \(T\in L(X,Y)\), dann ist der adj. Op. \(T':Y'\to X'\)
    def. durch \(T'y' = y' \circ T,\ \forall y'\in Y'\)
\end{definition}

\begin{definition}{Dualitätsklammer}
    Sei \(X\) ein norm. R. und \(X'\) sein Dualraum.
    Dann ist \(x'(x) = <x,x'>,\ \forall x\in X, x'\in X'\)
    die Dualitätsklammer.
\end{definition}

\begin{bemerkung}
    Für den adj. Op. gilt\\
    \(<x,T'y'> = <Tx,y'>\ 
    \forall x\in X,\forall y'\in Y'\)
\end{bemerkung}

\begin{bemerkung}
    \begin{enumerate}[label =(\roman*)]
        \item Die Abb. \(L(X,Y) \to L(X',Y'), T \mapsto T'\) ist lin., isom.,
            aber im alg. nicht surj.
        \item \(T\in L(X,Y), S\in L(Y,Z): (ST)'=T'S'\)
    \end{enumerate}
\end{bemerkung}

\begin{satz}{Zsmh. Kern und Bild Adj.}
    \(X,Y\) Br. \(T\in L(X,Y)\). Dann gilt
    \begin{enumerate}[label = (\roman*)]
        \item \(R(T)^\perp = N(T')\)
        \item \(\overline{R(T)} = N(T')_\perp\)
    \end{enumerate}
\end{satz}

\begin{korrolar}
    \(X,Y\) Br. \(T\in L(X,Y)\) mit abg. Bild.\\
    Die Gleichung \(Tx=y\) ist lösbar gdw. \(T'y' = 0\)
\end{korrolar}


\subsection{Bidualraum}

\begin{definition}{Bidualraum}
    \(X\) NR., \(X'' := (X')'\) heißt \textbf{Bidualraum}
\end{definition}

\begin{bemerkung}
    Für \(x\in X\) ist \(\delta_x: X' \to \KK, x'\mapsto <x,x'>\) lin. u. stetig
\end{bemerkung}

\begin{satz}{kanon. Einbettung}
    Die kanon. Einbettung \(J_X : X\to X'', x\mapsto \delta_x\) ist lin. u.
    isom. aber im alg. nicht surj.
\end{satz}

\begin{bemerkung}
    \begin{enumerate}[label=(\roman*)]
        \item Durch \(J_X\) wird \(X\) mit UR von \(X''\) identif.
        \item \(X\) voll. \(\implies J_X(X)\) voll.
        \item \(X\) norm. R., so ist \((\overline{J_X(X)}, J_X)\) Vervollstän.
        \item Jeder norm. R. ist isom. isomorph zu dichten UR eines BR.
    \end{enumerate}
\end{bemerkung}

\begin{lemma}
    \(X,Y\) BR, \(T\in L(X,Y)\). Dann ist \(T'' := (T')':X''\to Y''\) lin. u.
    stetig mit \(\|T''\| = \|T\|\) und \(T'' \circ J_X = J_Y \circ T \)
\end{lemma}

\begin{satz}{Satz von Schauder}
    \(X,Y\) BR., \(T\in L(X,Y)\). Dann gilt
    \(T\) komp. \(\Leftrightarrow T'\) komp.
\end{satz}


\subsection{Reflexivität}

\begin{definition}{Reflexivität}
    Ein BR \(X\) heißt reflexiv, falls die kanon. Einbettung \(J_X\) surj. ist.
\end{definition}

\begin{bemerkung}
    \begin{enumerate}[label = (\roman*)]
        \item Dann ist \(X\cong X''\)
        \item \(X\) nicht voll. \(\implies X\) nicht reflexiv
        \item \(p\in (1,\infty): l^p, L^p\) reflexiv
        \item \(c_0, l^1, l^\infty\) nicht reflexiv
        \item \(X\) end. dim. \(\implies X\) reflexiv
    \end{enumerate}
\end{bemerkung}

\begin{lemma}
    \begin{enumerate}[label = (\roman*)]
        \item Abg. UR von ref. R. sind ref.
        \item \(X\) ref \(\Leftrightarrow X'\) ref.
    \end{enumerate}
\end{lemma}

\begin{korrolar}
    \(X\) ref. ist seperabel \(\Leftrightarrow X'\) seperabel
\end{korrolar}

\begin{bemerkung}
    Hilberträume sind seperabel
\end{bemerkung}

\begin{definition}{schwache Konvergenz}
    \(X\) BR. \((x_n)_n) \in X\) heißt schw. konv. gegen \(x\in X\), falls
    \(<x_n,x'> \to <x,x'> \forall x'\in X'\)\\
    Not: \(x_n \rightharpoonup x, x_n \stackrel{\sigma}{\to} x,
    x_n \stackrel{\omega}{\to} x\)
\end{definition}

\begin{definition}{schwach* Konvergenz}
    \(X\) BR. \((x'_n)_n \in X'\) heißt schw.* konv. gegen \(x'\in X'\), falls
    \(<x,x'_n> \to <x,x'> \forall x\in X\)\\
    Not: \(x'_n \stackrel{*}{\rightharpoonup} x',
    x'_n \stackrel{\sigma^*}{\to} x', x'_n \stackrel{\omega^*}{\to}x'\)
\end{definition}

\begin{bemerkung}
    \begin{enumerate}
        \item Konv. in Norm \(\implies\) schw. Konv.
        \item schw./* GWe sind eindeutig
        \item schw. Konv. in \(X \implies\) schw.* Konv. in \(X''\)\\
            \(<x,x'>_{X\times X'} = \\ <x',J_X(x)>_{X' \times X''}\)
        \item Ist \(X\) ref., dann sind schw, Konv auf \(X'\) und schw.* Konv.
            auf \(X\) das Gleiche.
        \item \(X'\) ist schw.*-folgenvollständig. D.h. ist \(\forall x \in X:
            ( <x,x'_n>)_n\) CF in \(\KK\), dann ex. \(x' \in X'\) mit 
            \(x'_n \stackrel{*}{\rightharpoonup}x'\)
    \end{enumerate}
\end{bemerkung}

\begin{satz}{schw. Konv. auf dichter Menge}
    \(X\) BR, \(x_n,x\in X, x'_n,x'\in X'\)\\
    \(D\subset X: \overline{linD} = X\)\\
    \(D'\subset X': \overline{linD'} = X'\)\\
    Dann gilt
    \begin{enumerate}[label = (\alph*)]
        \item \(x_n \rightharpoonup x \Leftrightarrow \sup_n \|x_n\| < \infty\)
            und \(<x_n,x'> \to <x,x'> \forall x'\in D'\)
        \item \(x'_n \stackrel{*}{\rightharpoonup} x' \Leftrightarrow
            \sup_n \|x'_n\| < \infty\) und \(<x,x'_n> \to <x, x'>
            \forall x\in D\)
    \end{enumerate}
    In a) gilt \(\|x\| \leq \liminf \|x_n\|\)\\
    In b) gilt \(\|x'\| \leq \liminf \|x'_n\|\)
\end{satz}

\begin{satz}{Banach-Alaoglu Ver. I}
    Sei \(X\) seperabler BR.\\ Für \((x'_n)_n\) beschr. ex. \(x'\in X'\) und
    \((x'_{n_j})_j\) mit \(x_{n_j} \stackrel{*}{\rightharpoonup} x'\) und
    \(\|x'\| \leq \liminf \|x'_n\|\)
\end{satz}

\begin{bemerkung}
    \begin{enumerate}
        \item im alg. falsch für \(X\) nicht-seperabel
        \item Satz gilt nicht für schw. Konv.
    \end{enumerate}
\end{bemerkung}

\begin{satz}{Banach-Alaoglu Ver. II}
    Sei \(X\) reflexiver BR.\\ Für \((x_n)_n\) beschr. ex. \(x\in X\) und
    \((x_{n_j})_j\) mit \(x_{n_j} \rightharpoonup x\) und
    \(\|x\| \leq \liminf \|x_n\|\)
\end{satz}

\begin{satz}{Mazur}
    \(X\) BR, \(V\subset X\) abg.und konvex. Ist \((x_n)\) schw. konv. in V mit
    \(x_n \rightharpoonup x \implies x \in V\)
\end{satz}{Mazur}

\begin{satz}{Banach-Alaoglu + Mazur}
    \(X\) ref. \(\implies \overline{B_X}\) schw.-folgenkompakt
\end{satz}