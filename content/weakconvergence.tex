\section{Reflexivität und schwache Konvergenz}

\subsection{Adjungierter Operator}

\begin{definition}{Adj. Op.}
    Seien \(X,Y\) Br., \(T\in L(X,Y)\), dann ist der adj. Op. \(T':Y'\to X'\)
    def. durch \(T'y' = y' \circ T,\ \forall y'\in Y'\)
\end{definition}

\begin{definition}{Dualitätsklammer}
    Sei \(X\) ein norm. R. und \(X'\) sein Dualraum.
    Dann ist \(x'(x) = <x,x'>,\ \forall x\in X, x'\in X'\)
    die Dualitätsklammer.
\end{definition}

\begin{bemerkung}
    Für den adj. Op. gilt\\
    \(<x,T'y'> = <Tx,y'>\ 
    \forall x\in X,\forall y'\in Y'\)
\end{bemerkung}

\begin{satz}
    \(X,Y\) Br. \(T\in L(X,Y)\). Dann gilt
    \begin{enumerate}[label = (\roman*)]
        \item \(R(T)^\perp = N(T')\)
        \item \(\overline{R(T)} = N(T')_\perp\)
    \end{enumerate}
\end{satz}

\begin{korrolar}
    \(X,Y\) Br. \(T\in L(X,Y)\) mit abg. Bild.\\
    Die Gleichung \(Tx=y\) ist lösbar gdw. \(T'y' = 0\)
\end{korrolar}

