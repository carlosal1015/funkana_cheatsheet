\section{Reflexivität und schwache Konvergenz}

\subsection{Adjungierter Operator}

\begin{definition}{Adj. Op.}
    Seien \(X,Y\) Br., \(T\in L(X,Y)\), dann ist der adj. Op. \(T':Y'\to X'\)
    def. durch \(T'y' = y' \circ T,\ \forall y'\in Y'\)
\end{definition}

\begin{definition}{Dualitätsklammer}
    Sei \(X\) ein norm. R. und \(X'\) sein Dualraum.
    Dann ist \(x'(x) = <x,x'>,\ \forall x\in X, x'\in X'\)
    die Dualitätsklammer.
\end{definition}

\begin{bemerkung}
    Für den adj. Op. gilt\\
    \(<x,T'y'> = <Tx,y'>\ 
    \forall x\in X,\forall y'\in Y'\)
\end{bemerkung}

\begin{bemerkung}
    \begin{enumerate}[label =(\roman*)]
        \item Die Abb. \(L(X,Y) \to L(X',Y'), T \mapsto T'\) ist lin., isom.,
            aber im alg. nicht surj.
        \item \(T\in L(X,Y), S\in L(Y,Z): (ST)'=T'S'\)
    \end{enumerate}
\end{bemerkung}

\begin{satz}{Zsmh. Kern und Bild Adj.}
    \(X,Y\) Br. \(T\in L(X,Y)\). Dann gilt
    \begin{enumerate}[label = (\roman*)]
        \item \(R(T)^\perp = N(T')\)
        \item \(\overline{R(T)} = N(T')_\perp\)
    \end{enumerate}
\end{satz}

\begin{korrolar}
    \(X,Y\) Br. \(T\in L(X,Y)\) mit abg. Bild.\\
    Die Gleichung \(Tx=y\) ist lösbar gdw. \(T'y' = 0\)
\end{korrolar}


\subsection{Bidualraum}

\begin{definition}{Bidualraum}
    \(X\) NR., \(X'' := (X')'\) heißt \textbf{Bidualraum}
\end{definition}

\begin{bemerkung}
    Für \(x\in X\) ist \(\delta_x: X' \to \KK, x'\mapsto <x,x'>\) lin. u. stetig
\end{bemerkung}

\begin{satz}
    Die kanon. Einbettung \(J_X : X\to X'', x\mapsto \delta_x\) ist lin. u.
    isom. aber im alg. nicht surj.
\end{satz}

\begin{bemerkung}
    \begin{enumerate}[label=(\roman*)]
        \item Durch \(J_X\) wird \(X\) mit UR von \(X''\) identif.
        \item \(X\) voll. \(\implies J_X(X)\) voll.
        \item \(X\) norm. R., so ist \((\overline{J_X(X)}, J_X)\) Vervollstän.
        \item Jeder norm. R. ist isom. isomorph zu dichten UR eines BR.
    \end{enumerate}
\end{bemerkung}

\begin{lemma}
    \(X,Y\) BR, \(T\in L(X,Y)\). Dann ist \(T'' := (T')':X''\to Y''\) lin. u.
    stetig mit \(\|T''\| = \|T\|\) und \(T'' \circ J_X = J_Y \circ T \)
\end{lemma}

\begin{satz}{Satz von Schauder}
    \(X,Y\) BR., \(T\in L(X,Y)\). Dann gilt
    \(T\) komp. \(\Leftrightarrow T'\) komp.
\end{satz}