\section{Hauptsätze}


\subsection{Bairescher Kategoriensatz}
  $(X,d)$ metr. R.

\begin{definition}{Durchmesser}
  $A \subset X,\diam A := \sup\{d(x,y), x,y\in A\} $
\end{definition}

\begin{lemma}
  $(X,d)$ voll., $(A_n) \subset X$ \textit{abg.} mit
  $\emptyset \neq A_{n+1} \subset A_n$ und $\diam A_n \to 0$
  $\implies \cap_{n=1}^\infty A_n \neq \emptyset$
\end{lemma}

\begin{satz}{Satz von Baire}
  $(X,d)$ voll., $(O_n) \subset X$ \textit{offen} und \textit{dicht}
  $\implies \cap_{n=1}^\infty O_n \neq \emptyset$
\end{satz}

\begin{definition}{Kategorien}
  \begin{enumerate}[label = (\roman*)]
    \item $M \subset X$ \textbf{nirgends dicht},
      falls $\inter \overline{A} = \emptyset$
    \item $M \subset X$ \textbf{1. Kategorie}, falls $\exists (M_n) \subset X$
      \textit{nirgends dicht}, $M = \cup_{n=1}^\infty M_n$
    \item $M \subset X$ \textbf{2. Kategorie}, falls nicht von
      \textit{1. Kategorie}
  \end{enumerate}
\end{definition}

\begin{korrolar}
  $(X,d)$ voll., $M \subset X$ \textit{1. Kategorie} $\implies M^C = X\\M$
  \textit{dicht} in $X$
\end{korrolar}

\begin{korrolar}
  Ein voll. metr. R. ist von \textit{2. Kategorie}:\\
  $X = \cup_{n=1}^\infty A_n, A_n$ \textit{abg.}
  $\implies \exists N \in \NN: \inter A_N \neq \emptyset$
\end{korrolar}


\subsection{Prinzip der glm. Beschränktheit}
$X,Y$ norm. R.

\begin{definition}{pkt., glm. Beschränktheit}
  $J \subset L(X,Y)$
  \begin{enumerate}[label = (\roman*)]
    \item J \textbf{pkt. Beschränkt},
      falls $\forall x\in X:\sup \{\|Tx\|:\ T \in J\} < \infty$
    \item J \textbf{glm. Beschränkt},
      falls $\sup \{\|T\|:\ T \in J\} < \infty$
  \end{enumerate}
\end{definition}

\begin{satz}{Prinzip der glm. Beschränktheit}
  $X$ \textit{voll.}, $J \subset L(X,Y)$ \textit{pkt. Beschränkt} $\implies J$
  \textit{glm. Beschränkt}
\end{satz}

\begin{satz}{Banach-Steinhaus}
  $X$ \textit{voll.}, $(T_n) \subset L(X,Y)$,
  falls $\forall x\in X: Tx:= lim_{n\to \infty} T_nx$ ex.
  $\implies T \in L(X,Y)$. u.
  $\|T\| \leq \lim_{n\to \infty} \inf \|T_n\| < \infty$
\end{satz}

\begin{korrolar}
  $X,Y$ \textit{voll.}, $(T_n) \subset L(X,Y)$. Es sind äq.
  \begin{enumerate}[label = (\roman*)]
    \item $\exists T \in L(X,Y): \forall x\in X Tx = \lim_{n\to\infty}T_nx$
    \item $\sup_{n \in \NN} \|T_n\| <\infty $ u. $\exists D\subset X$
      \textit{dicht} m. $\forall x\in D: (T_nx)$ \textit{CF}
  \end{enumerate}
\end{korrolar}

\begin{definition}{starke Konvergenz}
  $(T_n)\subset L(X,Y),\ T\in L(X,Y)$. $(T_n)$ \textbf{konvergiert stark}
  gegen T, falls $\forall x\in X: Tx = \lim_{n\to\infty}T_nx$
\end{definition}

\begin{bemerkung}
  \textit{Konvergenz Operatornorm} $\implies$ \textit{starke Konvergenz}
\end{bemerkung}


\subsection{Satz von der offenen Abbildung}
$X,Y$ metr. R.

\begin{definition}{offene Abbildung}
  $f:X\to Y$ \textbf{offen}, falls $\forall O \in X$ \textit{offen}
  $\implies f(O)$ \textit{offen}
\end{definition}

\begin{lemma}
  $X,Y$ norm. R., $T:X \to Y$ lin. Dann sind äq.
  \begin{enumerate}[label = (\roman*)]
    \item $T$ \textit{offen}
    \item $\exists \varepsilon > 0 : B(0,\varepsilon) \subset T(B(0,1))$
  \end{enumerate}
\end{lemma}

\begin{satz}{Satz von der offenen Abbildung}
  $X,Y$ voll. $T \in L(X,Y)$ \textit{surj.} $\implies T$ \textbf{offen}
\end{satz}

\begin{korrolar}
  $(X,\|\cdot\|),(X,|||\cdot|||)$ \textit{voll.}
  $\exists M > 0 \forall x\in X: \|x\| \leq M |||x|||
  \implies \|\cdot\|,|||\cdot|||$ \textbf{äq.}
\end{korrolar}

\subsection{Satz vom abgeschlossenen Graphen}
$X,Y$ norm. R.

\begin{definition}{Abgeschlossenen lin. Abbildung}
  $D\subset X$ UR, $T:D \to Y$ lin.\\
  $T$ heißt \textbf{agb.}, falls für $(x_n) \subset D, x_n \to x \in X$ u.
  $Tx_n \to y \in Y \implies x \in D,\ Tx = y $
\end{definition}

\begin{definition}{Graph}
  $D\subset X$ UR, $T:D \to Y$ lin.\\
  $gr(T)=\{(x,Tx), x\in D \subset X \times Y$
\end{definition}

\begin{bemerkung}
  \begin{enumerate}[label=(\roman*)]
    \item $gr(T)$ ist \textit{UR} von $X \times Y$
    \item $T$ \textit{abg.} $\Leftrightarrow gr(T)$ \textit{abg.} in
      $X \times Y$ bzgl. \textbf{Produktnorm}
      $|||\cdot||| = ||\cdot||_X + ||\cdot||_Y$
  \end{enumerate}
\end{bemerkung}

\begin{lemma}
  $X,Y$ \textit{voll.}, $D\subset X$ \textit{UR}, $T:D \to Y$ \textit{abg.}
  Dann
  \begin{enumerate}[label=(\roman*)]
    \item $(D,|||\cdot|||_G)$ \textit{voll}, \textbf{Graphennorm}
      $|||\cdot|||_G = ||\cdot||_X+||T\cdot||_Y$
    \item $T\in L((D,|||\cdot|||_G) ,Y)$
  \end{enumerate}
\end{lemma}

\begin{satz}{Satz vom abg. Graphen}
  $X,Y$ voll., $T:X \to Y$ \textit{lin.}, \textit{abg.}
  $\implies T\in L(X,Y)$
\end{satz}
