\section{Operatoren}


\subsection{Stetige lineare Operatoren}
$X,Y$ norm. R., $T:X\to Y$ linear\\
\begin{satz}{Stetigkeit von lin. Abb.}
  Es sind äquivalent
  \begin{enumerate}[label = (\roman*)]
    \item $T$ stetig
    \item $T$ stetig in $0$
    \item ($T$ beschränkt $\Leftrightarrow \exists C \geq 0 \forall x \in X:
      \|Tx\| \leq C\|x\|$)
    \item $T$ ist lipschitz
    \item $T$ ist glm. stetig
  \end{enumerate}
\end{satz}

\begin{definition}{Raum besch., lin. Operatoren}
  $L(X,Y)=\{ T:X \to Y|T \text{ lin. u. besch.} \}$
\end{definition}

\begin{definition}{Operatornorm}
  \begin{align*}
    \|T\| &= \sup \{ \frac{\|Tx\|}{\|x\|}| x\in X \setminus \{0\}\} \\
    &= \inf \{ c \geq 0|\ \|Tx\| \leq c \|x\| \forall x\in X \} \\
    &= \sup_{x\in X, \|x\| = 1}\|Tx\| \\
    &= \sup_{x\in X, \|x\| \leq 1}\|Tx\|
  \end{align*}
\end{definition}

\begin{definition}{Dualraum}
  $X' = L(X,\KK)$ \textbf{Dualraum} zu $X$
\end{definition}

\begin{satz}{$L(X,Y)$ vollständig?}
  $Y$ voll. $\implies L(X,Y)$ vollständig bzgl. Op.norm
\end{satz}

\begin{satz}{lineare Fortsetzung}
  $Y$ voll., $D \subset X$ dichter UR, $T \in L(D,Y)$.\\
  Dann ex. eind. Fortsetzung $\tilde{T} \in L(X,Y)$, mit $\|T\|=\|\tilde{T}\|$
\end{satz}

\begin{definition}{Kern, Bild, Einbettung,...}
  \begin{enumerate}[label = (\roman*)]
    \item $N(T) = \{x \in X: Tx =0\}$ \textbf{Kern}
    \item $R(T) = \{y = Tx: x \in X\}$ \textbf{Bild}
    \item \textit{injektiver} Oper. heißt \textbf{Einbettung}
      $X \leftrightarrow Y$
    \item \textit{bijektiver} Oper. mit \textit{stetiger Inverser} $T^{-1}$
     heißt \textbf{Isomorphismus} $X \simeq Y$
    \item Oper. mit $\|Tx\| = \|x\|$ heißt \textbf{isometrisch}
    \item Oper. mit $\|T\| \leq 1$ \textbf(!!!) heißt \textbf{kontraktiv}
  \end{enumerate}
\end{definition}

\begin{bemerkung}
  \begin{itemize}
    \item $N(T)$ \textit{abgeschlossen}, ($T\in L(X,Y)$)
    \item $T$ \textit{isometrisch} $\implies T$ \textit{kontraktiv} u.
      \textit{injektiv}
    \item $T$ \textit{kontraktiv} $\implies T$ \textit{stetig}
    \item $T$ \textit{isometrisch} $\implies T^{-1}:R(T) \to X$
      \textit{isometrisch}
    \item $R(T)$ \textit{abgeschlossen in} $Y$
  \end{itemize}
\end{bemerkung}


\subsection{Summen norm. R.}
  $X$ norm. R.

\begin{definition}{Projektion}
  $P \in L(X), P^2=P$ heißt \textbf{Projektion}
\end{definition}

\begin{lemma}
  $P \in L(X)$ Projektion, $X_1 = N(P) = R(1-P),\ X_2 = R(P) = N(1-P)$.
  Dann ist $(1-P)$ auch Projektion und $X = X_1 \oplus X_2$ sowie
  $\|P\| \geq 1, P \neq 0$
\end{lemma}

\begin{lemma}
  $X = X_1 \oplus X_2$. Dann $\exists P \in L(X)$ ein. Projektion mit
  $R(P) = X_1, N(P) = X_2$
\end{lemma}

\begin{bemerkung}
  $X$ voll. $\implies (X_1 \oplus X_2 \simeq X_1 \times X_2)$
\end{bemerkung}


\subsection{Quotientenräume}
$X norm. R.$

\begin{definition}{Quotientenraum}
  $Y$ UR von $X$.\\
  $X/Y = \{\hat{x} = x + Y| x\in X\}$ \textbf{Quotientenraum}
\end{definition}

\begin{definition}{Quotientenabbildung}
  $Q:X \to X/Y,\ Qx = \hat{x}$ \textbf{Quotientenabbildung},
  \textit{linear} u. \textit{surjektiv}, $N(Q) = Y$
\end{definition}

\begin{definition}{Quotientennorm}
  $\|\hat{x}\| = \inf\{\|x-y\|\ |y\in Y \} (= d(x,Y))$
\end{definition}

\begin{satz}{$X/Y$ norm. R.?}
  $Y$ \textbf{\uline{abg.}} UR von $X$.\\
  Dann ist $X/Y$ mit der Quo.norm ein norm. R. Ist $X$ \textbf{voll.}, dann
  auch $X/Y$
\end{satz}


\subsection{Kompakte Operatoren}

$X,Y$ norm. R., $B_X=\{x \in X|\ \|x\| \leq 1\}$, $T:X \to Y$ linear

\begin{definition}{kompakter Operator}
  $T$ \textbf{kompakt}, falls $T(B_X)$ \textit{relativ kompakt} in $Y$
\end{definition}

\begin{bemerkung}
  \begin{enumerate}[label = (\roman*)]
    \item $T$ \textit{kompakt} $\implies T \in L(X,Y)$
    \item $Y$ voll. Dann gilt $T$ \textit{kompakt}
      \begin{enumerate}[label = $\Leftrightarrow$]
        \item $T(B_X)$ \textit{totalbeschränkt}
        \item für $(x_n)$ \textit{beschränkt} enthält
          $(Tx_n)$ \textit{konv. TF}
      \end{enumerate}
  \end{enumerate}
\end{bemerkung}

\begin{satz}{kompakte Operatoren}
  $X,Y,Z$ voll.
  \begin{enumerate}[label=(\roman*)]
    \item $K(X,Y)$ \textit{abg. UR} von $L(X,Y)$ und voll.
    \item $T\in L(X,Y),\ S\in L(Y,Z)$, ist $T \text{ o. } S$ kompakt
      $\implies ST \in K(X,Z)$
    \item $T \in L(X,Y),\ dimR(T) < \infty \implies T \in K(X,Y)$
    \item $dimX < \infty \implies T \in K(X,Y)$
  \end{enumerate}
\end{satz}

\begin{korrolar}
  $X,Y$ \textit{voll.}, $T \in L(X,Y)$. Es gilt
  $\exists (T_n)_n \subset L(X,Y), dimR(T_n)<\infty, \|T_n-T\| \to 0
  \implies T \in K(X,Y)$
\end{korrolar}
