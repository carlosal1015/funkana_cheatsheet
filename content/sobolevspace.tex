\section{Sobolevräume}

\(\Omega \subset \RR^n \text{ offen}\)\\
\(L_{loc}^1(\Omega) = \{f:\Omega \to \KK \text{ mb: }
f\indicator_K \in L^1(\Omega),\ K\subset \Omega \text{ kpt.}\}\) \\
\(\supp\varphi = \{x\in \Omega: \varphi(x) \neq 0\}\)\\
\(C_c^\infty(\Omega)=\{\varphi:\Omega \to \KK \in C^\infty: 
\supp\varphi\subset\Omega \text{ kpt.}\}\)\\

\begin{definition}{schwache Ableitung}
    \(\Omega \subset \RR^n\) Gebiet,
    \(f\in L^1_{lov}(\Omega), j\in \{1,..,n\}\)\\
    Existiert \(g \in L^1_{loc}(\Omega):\)
    \(
        \int_\Omega f\cdot \partial_j\varphi dx =
         - \int_\Omega g\cdot \varphi dxm, \ \ 
         \forall \varphi\in C_c^{\infty}(\Omega),
    \)
    dann heißt \(f\) schwach nach \(x_j\) ableitbar, und \(\partial_jf=g\) ist
    die \textbf{schwache Ableitung} in Richtung \(x_j\).

\end{definition}

\begin{definition}{Faltung}
    Für \(f,g:\RR^n \to \KK\) ist
    \(
        (f\ast g )(x) = \int_{\RR^n}f(y)g(x-y)dy
        = \int_{\RR^n}f(x-y)g(y)dy
    \) 
    die Faltung von \(f\) und \(g\).
\end{definition}

\begin{lemma}
    \(f\in L_{loc}^1(\RR^n),\ \varphi\in C_c^{\infty}(\RR^n), 
    \alpha \in \NN_0^n\)
    \begin{enumerate}[label = (\roman*)]
        \item \(f\ast \varphi \in C^\infty(R^n),\ 
            \partial^\alpha(f\ast \varphi) = f\ast \partial^\alpha\varphi\)
        \item \(p\in [1,\infty),\ f\in L^p(\RR^n)\)\\
            \(\supp\varphi \subset B(0,1)\)\\
            \(\int_{\RR^n}\varphi(x)dx = 1\)\\
            \(\varphi_k(x):=k^n\varphi(kx), k\in\NN\)\\
            Dann ist \(f\ast \varphi_k \in L^p(\RR^n)\) und\\
            \(\|f\ast \varphi_k -f\|\to 0\)
    \end{enumerate}
\end{lemma}

\begin{korrolar}
    Ist \(g\in L_{loc}^1(\Omega)\) mit \(\int_\Omega g\varphi dx = 0,
    \forall \varphi \in C_c^\infty(\Omega)\), dann ist \(g=0\) f.ü.
\end{korrolar}
    
\begin{definition}{Sobolevraum}
    \(\Omega \subset \RR^n\) Gebiet, \(p \in [1,\infty]\)
    \begin{enumerate}[label=(\roman*)]
        \item \(W^{1,p}(\Omega) = \{
            f\in L^p(\Omega):\partial_jf\text{ ex. u. } \partial_j f
            \in L^p(\Omega),\forall j\in\{1\dots,n\}\})\) heißt der Sobolevraum
            1. Ordnung
        \item \(W^{m,p}(\Omega) = \{f\in L^p(\Omega):\partial^\alpha f 
            \text{ ex. u. } \partial^\alpha f \in L^p(\Omega),
            \forall\alpha\in\NN_0^n, |\alpha|\leq m\}\) heißt der Sobolevraum
            m. Ordnung
    \end{enumerate}
\end{definition}

\begin{definition}{Norm - SR. 1. Ordnung}

    \begin{enumerate}[label=(\roman*)]
        \item \(p\in [1,\infty)\):\\
            \( \|f\|_{W^{1,p}} = (\|f\|_p^p +
            \sum_{j=1}^{n}\|\partial_jf\|_p^p)^{\frac{1}{p}} \)
        \item \(p = \inf\):\\
            \( \|f\|_{W^{1,p}} =
            \max\{\|f\|_\infty, \|\partial_1f\|_\infty, \dots,
            \|\partial_nf\|_\infty\} \)
    \end{enumerate}
\end{definition}

\begin{satz}{Vollständige Sobolevräume}
    Für \(m \in \NN,\ p\in [1,\infty]\) ist
    \((W^{m,p}(\Omega),\|\cdot\|_{W^{m,p}})\) ein Banachraum.\\
    Weiter ist \(W^{m,2}(\Omega)=H^m(\Omega)\) ist Hilbertraum mit Skalarprodukt
    \((f,g)_{H^m(\Omega)}=\sum_{|\alpha|\leq m}
    (\partial^\alpha f,\partial^\alpha g)_{L^2(\Omega)}\)
\end{satz}

\begin{lemma}
    Sei \(\Omega \subset \RR^n\) Gebiet, \(f\in L_{loc}^1(\Omega)\) mit
    \(\partial_jf\in L_{loc}^1(\Omega), j=1,\dots,n,\ g\in
    C^\infty(\Omega)\).
    Dann ex. \(\partial_j(fg), j=1,\dots,n\) in \(\Omega\) und 
    \(\partial_j(fg)= (\partial_jf)g + f(\partial_jg), j=1,\dots,n\)
\end{lemma}

\begin{bemerkung}
    Für \(p\in [1,\infty], f\in W^{1,p}(\Omega),\ 
    g\in W^{1,p'}(\Omega)\) gilt \(fg\in W^{1,1}(\Omega)\) und
    \(\partial_j(fg)= (\partial_jf)g + f(\partial_jg), j=1,\dots,n\)
\end{bemerkung}

\begin{bemerkung}
    Für \(f\in L_{loc}^1(\RR^n),\ \alpha \in \NN_0^n\) ex.
    \(\partial^\alpha f\) in \(\RR^n\), dann gilt
    \(\forall\varphi\in C_c^\infty:\ \partial^\alpha(f\ast \varphi) = 
    \partial^\alpha f \ast \varphi\) 
\end{bemerkung}

\begin{satz}{Dicht im SR}
    Für \(p\in [1,\infty),\ m\in\NN\) ist \(C_c^\infty(\RR^n)\) dicht in
    \(W^{m,p}(\RR^n)\)
\end{satz}


