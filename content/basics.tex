
\section{Räume}
\begin{definition}{Metrik}
  $d:X \times X \to [0,\infty)$, $x,y \in X$
  \begin{enumerate}[label=(M\arabic*)]
    \item $d(x,y) = 0 \Leftrightarrow x = y$
    \item $d(x,y) = d(y,x)$
    \item $d(x,z) \leq d(x,y) + d(y,z)$
  \end{enumerate}
\end{definition}

Genügt $d$ nur (M2), (M3) und ist $d(x,x) = 0$, dann heißt $d$
\textbf{Halbmetrik}.
\heel

\begin{definition}{Metrischer Raum}
  $X \neq \emptyset, d$ Metrik, dann heißt $(X,d)$ \textit{metrischer Raum}
\end{definition}

\begin{definition}{Vollständigkeit}
  Ein \textbf{MR} $(X,d)$ heißt \textit{vollständig}, wenn jede CF
  konvergiert.
\end{definition}

\begin{definition}{Norm}
  $\norm{\cdot}:X\to\RR$, $X\ \KVR$, $x,y \in X, \lambda \in \KK$
  \begin{enumerate}[label=(N\arabic*)]
    \item $\norm{x} = 0 \implies x = 0$
    \item $\norm{\lambda x} = |\lambda|\norm{x}$
    \item $\norm{x+y} \leq \norm{x} + \norm{y}$
  \end{enumerate}
\end{definition}

Genügt $\norm{\cdot}$ nur (N2), (N3), dann heißt $\norm{\cdot}$
Halbnorm auf X.
\heel

\begin{definition}{Normierter Raum}
  $X\ \KVR, \norm{\cdot}$ Norm, dann heißt $(X,\norm{\cdot})$
  \textit{normierter Raum}.
\end{definition}

\begin{definition}{Banachraum}
  Ein vollständiger NR ${(X,\norm{\cdot})}$ heißt Banachraum
\end{definition}

\begin{satz}{Voll. NR u. Reihen}
  Sei \((X,\|\cdot\|)\) NR. Es sind äq.
  \begin{enumerate}[label=(\roman*)]
    \item \((X,\|\cdot\|)\) voll.
    \item Für \(\sum_{n=1}^\infty \|x_n\| < \infty\) ex.
      \(x\in X\) mit \(lim_{N\to\infty}\|x-
      \sum_{n=1}^N x_n\| = 0\)
  \end{enumerate}
\end{satz}

\begin{satz}{CF-TF Konvergenz}
  Sei \((x_n)_n\) CF in \((X,d)\) MR. Gibt es eine TF
  \((x_{n_j})_j\), die gegen \(x\in X\) konvergiert,
  so konvergiert \((x_n)_n\) gegen \(x\).
\end{satz}

\begin{satz}{TF Konvergenz}
  Sei \((x_n)_n\) eine Folge in \((X,d)\) MR. Dann sind
  äq.
  \begin{enumerate}[label=(\roman*)]
    \item \((x_n)_n\) konvergiert gegen \(x\)
    \item Jede TF \((x_{n_j})_j\) hat eine gegen \(x\)
      konvergente TF
  \end{enumerate}
\end{satz}