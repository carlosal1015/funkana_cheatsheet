\section{Topologie}

\begin{definition}{Offen und Abgeschlossen}
  $M \subset X$ \textbf{offen} $\Leftrightarrow \forall x_0 \in M
  \exists r>0: B(x_0,r) \subset M$\\
  $M \subset X$ \textbf{abgeschlossen} $\Leftrightarrow X\setminus M$ offen.
\end{definition}

\begin{satz}{Offenheit mit Folgen}
  Sei \(U\subset X\). Es sind äq.
  \begin{enumerate}[label=(\roman*)]
    \item \(U\) \textit{offen}
    \item \(\forall x\in U\) und jede gegen \(x\) konv.
      Folge \((x_n)_\ninNN\), ex \(N\in\NN\) mit \(x_n \in U,
      \forall n\geq N\)
  \end{enumerate}
\end{satz}

\begin{lemma}
  \begin{enumerate}[label = (\arabic*)]
    \item Die Vereinigung \textit{beliebig} vieler offener Mengen ist offen.
    \item Der Schnitt \textit{endlich} vieler offener Mengen ist offen.
  \end{enumerate}
\end{lemma}

\begin{definition}{innerer Punkt, Randpunkt, Abschluss}
  $M \subset X$
  \begin{enumerate}[label = (\arabic*)]
    \item $x_0 \in M$ \textbf{innerer Pkt.} von M
      $\Leftrightarrow \exists r>0: B(x_0,r) \subset M$
    \item $int(M)$ Menge aller \textit{inneren Pkt.}
    \item $x_0 \in X$ \textbf{Randpunkt} von M $\Leftrightarrow
      \forall r>0: B(x_0,r) \cap M \neq \emptyset \wedge B(x_0,r)
      \cap (X\setminus M) \neq \emptyset.$
    \item \textbf{Rand} $\partial M$ Menge aller \textit{Randpkt.}
    \item \textbf{Abschluss} $\overline{M} = \{x \in X|\forall r>0 B(x,r)\cap
      M \neq \emptyset$
  \end{enumerate}
\end{definition}

\begin{itemize}
  \item $int(X\setminus M) = X \setminus \overline{M}$
  \item $\overline{X \setminus M} = X \setminus int(M)$
  \item $M$ offen $\Leftrightarrow int(M) = M$
  \item $M$ abg. $\Leftrightarrow \overline{M} = M$
  \item $\partial M = \overline{M} \cap \overline{X\setminus M}$
  \item $\overline{M} = int(M) \discup \partial M$
  \item $X = int(M) \discup \partial M \discup int(X \setminus M)$
\end{itemize}
\heel

\begin{lemma}
  $M \subset X, x_0 \in X\\ x_0 \in \overline{M} \Leftrightarrow
  \exists (x_n)_n \subset M: x_n \to x_0$
\end{lemma}

\begin{korrolar}
  $M \subset X \text{ abg.} \Leftrightarrow \forall (x_n)_n \subset M:
  {(x_n \to x_0 \implies x_0 \in M)}$
\end{korrolar}

\begin{satz}{voll. Teilraum}
  $(X,d)$ metr., \textit{voll.}, $\emptyset \neq M \subset X$\\
  $(M,d_{|M})$ \textit{voll.} $\Leftrightarrow M$ abg. in $X$.
\end{satz}

\dheel

\begin{definition}{Dicht, seperabel}
  $(X,d)$ metr., $D,M \subset X$
  \begin{enumerate}[label = (\arabic*)]
    \item $D$ \textbf{dicht} in $M \Leftrightarrow M \subset \overline{D}$
    \item $(X,d)$ \textbf{seperabel} $\Leftrightarrow \exists A \subset X:
    A \text{ abz. und dicht in }X$
  \end{enumerate}
\end{definition}

\begin{korrolar}
  $D$ \textbf{dicht} in $M \Leftrightarrow \forall x
  \in M \exists (x_n)_n \subset D: x_n \to x$
\end{korrolar}

\begin{lemma}
  $(X, \norm{\cdot})$ ist \textbf{seperabel} $\Leftrightarrow
  \exists A \subset X$ abz. u. $lin(A)$ dicht in \(X\)
\end{lemma}



\subsection{Teilraumtopologie}
\((X,d)\) MR, \(M\subset X, d_M = d_{|M\times M}\)\\

\begin{definition}{relativ offen}
  $A \subset M$ heißt \textbf{relativ offen}, falls $A$ offen in $(M,d_M)$
\end{definition}

\begin{lemma}
  \(A\subset M\) ist offen in \(M\), gdw. es \(A'\subset X\) offen gibt, mit
  \(A = A'\cap X\)
\end{lemma}

\begin{korrolar}
  Sei \(A \subset M\). Für den Abschluss \(\overline{A}^M\) in \(M\) gilt
  \(\overline{A}^M = \overline{A}\cap M\)
\end{korrolar}

\begin{satz}{Seperabler Teilraum}
  Sei \(X\) seperabel \(\implies M\) seperabel.
\end{satz}

\subsection{Stetigkeit}

$X,Y$ metr. R., $f:X\to Y$\\
\begin{definition}{Stetigkeit}
  \begin{enumerate}[label = (\arabic*)]
    \item  f heißt \textbf{stetig}, falls\\
      $\forall O \subset Y \text{ offen } \implies f^{-1}(O) \text{ offen }$
    \item f heißt \textbf{stetig in} $x_0 \in X$, falls\\
      $\forall V \subset Y \text{ Umgebung v. } f(x_0) \exists
      U \subset X \text{ Umgebung v. } x_0: f(U) \subset V$
  \end{enumerate}
\end{definition}

\begin{bemerkung}
  f stetig in $x_0 \in X$
  \begin{enumerate}[label= $\Leftrightarrow$]
    \item $\forall \varepsilon >0 \exists \delta >0 \forall x \in X:
      d(x,x_0)<\delta \implies d(f(x),f(x_0)) < \varepsilon$
    \item $\forall \varepsilon >0 \exists \delta >0:
      B(x_0,\delta) \subset f^{-1}(B(f(x_0)), \varepsilon)$
    \item $x_n \to x_0 \implies f(x_n) \to f(x_0)$
  \end{enumerate}
\end{bemerkung}

\begin{definition}{glm. Stetigkeit}
  $f$ heißt \textbf{glm. Stetig}, falls
  $\forall \varepsilon >0 \exists \delta >0 \forall x,y \in X:
  d(x,y) < \delta \implies d(f(x),f(y)) < \varepsilon$
\end{definition}

\begin{bemerkung}
    $f$ glm stetig, $(x_n)$ CF $\implies (f(x_n))$ CF
\end{bemerkung}

\begin{definition}{Lipschitz-Stetig}
  $f$ heißt \textbf{Lipschitz-Stetig}, falls
  $\exists L >0 \forall \delta >0 \forall x,y \in X:
  d(f(x),f(y)) \leq L d(x,y)$
\end{definition}

\begin{satz}{glm. Fortsetzung}
  $D\subset X,\ X,Y$ vollständig, $f$ glm. Stetig.\\
  Dann existiert eine \textit{eindeutige, glm. Stetige} Fortsetzung
  $\tilde{f}: \overline{D} \to Y$
\end{satz}



\subsection{Äquivalente Normen}

$(\|\cdot\|, X), (\||\cdot\||, X)$ normierte Räume\\

\begin{definition}{Äquivalente Normen}
  $\|\cdot\|, \||\cdot\||$ sind äquivalent, falls
  $\exists c,C>0 \forall x\in X: c\|x\| \leq \||x\|| \leq C \|x\|$
\end{definition}

\begin{satz}{äq. Normen und GWe}
  \begin{enumerate}[label = $\Leftrightarrow$]
    \item $\|\cdot\|, \||\cdot\||$ äquivalent
    \item $\|\cdot\|, \||\cdot\||$ haben die \textit{gleichen konv. Folgen}
    \item $\|\cdot\|, \||\cdot\||$ haben die \textit{gleichen Nullfolgen}
  \end{enumerate}
\end{satz}

\begin{bemerkung}

  \begin{enumerate}[label = (\roman*)]
    \item $(X,d_1),(X,d_2)$ metr. Räume mit den selben konv. Folgen
      $\not\Rightarrow$ dieselben CF
    \item Sind $\|\cdot\|, \||\cdot\||$ äquivalent, dann haben die Räume die
      \textit{gleichen} offenen (abgl.) Mengen
    \item Sind $\|\cdot\|, \||\cdot\||$ äquivalent, dann gilt
      $(X,\|\cdot\|)$ voll. $\Leftrightarrow$ $(X,\||\cdot\||)$ voll.
    \item In end. dim. Räumen sind alle Normen äq.
  \end{enumerate}
\end{bemerkung}



\subsection{Kompaktheit}

\begin{definition}{Komapkt, folgenkom., rel. kom., totalbeschränkt}
  $(X,d)$ metr. R., $K \subset X$
  \begin{enumerate}[label = (\roman*)]
    \item $K$ \textbf{kompakt}, falls jede offenen Überdeckung von K eine
      endliche Teilüberdeckung besitzt
    \item $K$ \textbf{folgenkompakt}, falls jede Folge in K eine konvergente TF
      besitzt
    \item $K$ \textbf{relativ kompakt}, falls $\overline{K}$ kompakt
    \item $K$ \textbf{totalbeschränkt}, falls
      $\forall \varepsilon >0 \exists x_1,...,x_m \in X:
      K \subset \cup_{i=1}^{m}B(x_j,\varepsilon) $
  \end{enumerate}
\end{definition}

\begin{satz}{Kompaktheit in metr. R.}
  \begin{enumerate}[label = $\Leftrightarrow$]
    \item $K$ \textit{kompakt}
    \item $K$ \textit{folgenkompakt}
    \item $K$ \textit{totalbeschränkt} und \textit{vollständig}
  \end{enumerate}
\end{satz}

\begin{bemerkung}
  \begin{enumerate}[label=(\roman*)]
    \item $K$ \textit{rel. kompakt} u. \textit{abgeschlossen}
      $\implies$ $K$ \textit{kompakt}
    \item $K$ \textit{komapkt} $\implies$ $K$ \textit{totalbeschränkt}
    \item $K$ \textit{totalbeschränkt}, dann können die $x_j$ in $K$
      gewählt werden
    \item $K$ \textit{totalbeschränkt} $\implies K$ \textit{beschränkt}
    \item $K$ voll. $\implies K$ \textit{abgeschlossen} in X
    \item $K$ \textit{kompakt} $\implies K$ \textit{abgeschlossen} u.
      \textit{beschränkt}
    \item $K$ \textit{totalbeschränkt} $\implies K$
      \textit{seperabel}
    \item $K$ \textit{komapkt} $A \subset K \implies A$ \textit{rel. kompakt}
  \end{enumerate}
\end{bemerkung}

\begin{satz}{Kompaktheit im endlichen}
  $(X, \|\cdot\|)$ normierter Raum. Dann ist äquivalent
  \begin{enumerate}[label = (\roman*)]
    \item $dim\ X < \infty$
    \item $B_X = \{x\in X: \|x\| \leq 1\} = \overline{B}(0,1)$ ist kompakt
    \item Jede \textit{abgeschlossene} u. \textit{beschränkte} Teilmenge ist
      \textit{kompakt}
  \end{enumerate}
\end{satz}

\begin{satz}{Rieszsches Lemma}
  $(X, \|\cdot\|)$ normierter Raum, $Y \neq X$ abg. UR, $\delta>0$\\
  Dann $\exists x_{\delta} \in X, \|x_{\delta}\| = 1,  \|x_{\delta}-y\|
  \geq 1-\delta \forall y \in Y$
\end{satz}

\begin{lemma}
  $(X, \|\cdot\|)$ normierter Raum, $Y$ end. dim. UR.\\
  Dann ist $Y$ abg. in $X$
\end{lemma}

\begin{satz}{Stetige Funktion auf Kompaktum}
  $X,Y$ metr. R., $X$ \textit{kompakt}, $f:X\to Y$ stetig.\\
  Dann ist $f$ glm stetig und $f(X)$ kompakt
\end{satz}

\begin{definition}{Gleichgradig Stetig}
  $(K,d)$ metr. Raum, $K$ kompakt, $M \subset C(K)$\\
  $M$ heißt gleichgradig Stetig, falls
  $\forall \varepsilon >0 \exists \delta >0 \forall f \in M \forall s,t \in K:
  (d(s,t) < \varepsilon \implies |f(s) - f(t)| < \varepsilon)$
\end{definition}

\begin{satz}{Arzela-Ascoli}
  $(K,d)$ metr. Raum, $K$ kompakt, $M \subset C(K)$\\
  $M$ \textit{beschränkt} u \textit{gleichgradig Stetig} $\implies$ $M$
  \textit{rel. Komapkt}
\end{satz}
